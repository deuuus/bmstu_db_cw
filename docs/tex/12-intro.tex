\chapter*{ВВЕДЕНИЕ}
\addcontentsline{toc}{chapter}{ВВЕДЕНИЕ}

В современном мире люди генерируют огромное количество данных. С 2012 года и по настоящее время ежедневно генерируется около 2.5 $\cdot 10^{18}$ байтов информации\cite{stat}. Объемы производимых данных постоянно увеличиваются как следствие автоматизации многих процессов жизнедеятельности человека, также рост объема данных обусловлен тем фактом, что население планеты постоянно растет.

В данной курсовой работе будет рассмотрена сфера здравоохранения человека и генерируемая информация, связанная с ней. 

Целью курсовой работы является разработка и реализация медицинской информационной системы (далее МИС), прикладные команды которой используют данные, хранящиеся в базе данных, для автоматизации и анализа деятельности клиники с целью дальнейшего планирования деятельности учреждения. Исходными данными для разработки являются данные о пациентах, врачах, заболеваниях, методах их лечения и др.

Выбор предметной области для выполнения курсовой работы был обусловлен следующими причинами:

\begin{enumerate}
	\item Согласно Федеральному закону от 29 июля 2017 г. №242-ФЗ <<О внесении изменений в отдельные законодательные акты РФ по вопросам применения информационных технологий в сфере охраны здоровья>> было приказано <<создавать медицинские информационные системы, содержащие данные о пациентах, об оказываемой им медицинской помощи, о медицинской деятельности медицинских организаций с соблюдением требований, установленных законодательством РФ в области персональных данных, и соблюдением врачебной тайны>>. Таким образом, создание МИС теперь обусловлено на законодательном уровне.
	\item Одним из важных последствий развития пандемии COVID-19 стал резкий рост обращений в больницы и оказания соответствующих услуг, в том числе вакцинации населения, возникла острая необходимость в еще более стремительной автоматизации данного процесса.
\end{enumerate}

%Оценка состояния!!!

\clearpage

Для достижения поставленной цели необходимо выполнить следующие задачи:

\begin{itemize}
	\item формализовать поставленную задачу;
	\item описать структуру базы данных: сущности, связи между ними, основные понятия;
	\item провести анализ моделей данных и выбрать подходящий вариант для решения задачи;
	\item спроектировать и заполнить базу данных, реализовать соответствующие запросы;
	\item реализовать интерфейс для доступа к базе данных;
	\item реализовать программное обеспечение, которое позволит пользователю создавать, получать и изменять сведения из разработанной базы данных.
\end{itemize}

