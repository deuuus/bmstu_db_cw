\begin{appendices}
\chapter{Сценарий создания базы данных}

На листингах А.1-А.2 представлен сценарий создания базы данных.

\begin{lstlisting}[caption={Сценарий создания базы данных}]
CREATE TABLE patients (
  id                 serial    not null primary key,
  name               varchar   not null,
  surname            varchar   not null,
  gender             varchar   not null,
  birth_year         integer   not null,
  phone              varchar   not null,
  email              varchar   not null unique,
  encrypted_password varchar   not null,
  check (birth_year between 1922 AND 2022)
);

CREATE TABLE specializations (
  id     serial  not null primary key,
  name   varchar not null,
  salary integer not null,
  check (salary > 0)
);

CREATE TABLE diseases (
  id      serial not null primary key,
  name    varchar not null,
  spec_id integer   references specializations (id)
);

CREATE TABLE medicines (
  id   serial  not null primary key,
  name varchar not null,
  cost integer not null,
  check (cost > 0)
);

\end{lstlisting}
\clearpage
\begin{lstlisting}[caption={Сценарий создания базы данных(продолжение)}]
CREATE TABLE doctors (
  id                 serial    not null   primary key,
  name               varchar   not null,
  surname            varchar   not null,
  work_since         integer   not null,
  spec_id            integer   references specializations (id),
  email              varchar   not null unique,
  encrypted_password varchar   not null,
  check (work_since between 1942 AND 2022)
);

CREATE TABLE visits (
  id          serial  not null primary key,
  status      varchar not null,
  patient_id  integer references patients (id),
  doctor_id   integer references doctors (id),
  check (status in ('Done', 'Active'))
);

CREATE TABLE records (
  id          serial  not null primary key,
  visit_id    integer references visits (id),
  disease_id     integer references diseases (id),
  medicine_id    integer references medicines (id)
);

CREATE TABLE admins (
  id                 serial    not null primary key,
  name               varchar   not null,
  surname            varchar   not null,
  email              varchar   not null unique,
  encrypted_password varchar   not null
);

CREATE TABLE ops_stat
(
  operation char(1) not null,
  date timestamp not null,
  user_id text not null
);
\end{lstlisting}
\clearpage
На листинге А.3 представлен сценарий создания ролей и выделения прав.

\begin{lstlisting}[caption={Создание ролей и выделение прав}]
create role patient with login password '111';
grant select, insert on table patients to patient;
grant select,insert on table visits to patient;
grant select on table records to patient;

create role doctor with login password '222';
grant select, insert on table doctors to doctor;
grant select on table medicines to doctor;
grant select on table diseases to doctor;
grant select, update on table visits to doctor;
grant select, insert on table records to doctor;

create role admin with login password '333';
grant select, insert on table admins to admin;
grant select on table patient to admin;
grant select on table medicines to doctor;
grant select on table diseases to doctor;
grant select, insert on table doctors to admin;
grant select, insert on table records to admin;
\end{lstlisting}

\clearpage
На листинге А.4 представлен сценарий создания логирующего триггера.

\begin{lstlisting}[caption={Сценарий создания логирующего триггера}]
CREATE OR REPLACE FUNCTION logging()
RETURNS TRIGGER
AS
$code$
	BEGIN
	IF (TG_OP = 'INSERT') THEN
		INSERT INTO ops_stat SELECT 'I', now(), user;
	ELSIF (TG_OP = 'DELETE') THEN
		INSERT INTO ops_stat SELECT 'S', now(), user;
	ELSIF (TG_OP = 'UPDATE') THEN
		INSERT INTO ops_stat SELECT 'S', now(), user;
	END IF;
	RETURN NULL;
	END;
$code$
LANGUAGE PLPGSQL;

CREATE TRIGGER log AFTER INSERT OR DELETE OR UPDATE ON patients
FOR ROW EXECUTE FUNCTION logging();

CREATE TRIGGER log AFTER INSERT OR DELETE OR UPDATE ON doctors
FOR ROW EXECUTE FUNCTION logging();

CREATE TRIGGER log AFTER INSERT OR DELETE OR UPDATE ON admins
FOR ROW EXECUTE FUNCTION logging();

CREATE TRIGGER log AFTER INSERT OR DELETE OR UPDATE ON diseases
FOR ROW EXECUTE FUNCTION logging();

CREATE TRIGGER log AFTER INSERT OR DELETE OR UPDATE ON specializations
FOR ROW EXECUTE FUNCTION logging();

CREATE TRIGGER log AFTER INSERT OR DELETE OR UPDATE ON medicines
FOR ROW EXECUTE FUNCTION logging();

CREATE TRIGGER log AFTER INSERT OR DELETE OR UPDATE ON visits
FOR ROW EXECUTE FUNCTION logging();

CREATE TRIGGER log AFTER INSERT OR DELETE OR UPDATE ON records
FOR ROW EXECUTE FUNCTION logging();
\end{lstlisting}

\end{appendices}